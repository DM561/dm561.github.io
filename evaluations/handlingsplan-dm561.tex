% latex handlingsplan template v0.001, daniel
\documentclass[11pt]{article}
\usepackage[a4paper, total={7in, 9in}]{geometry}
\usepackage{helvet}
\renewcommand{\familydefault}{\sfdefault}
\usepackage[UKenglish]{babel}
\usepackage[utf8]{inputenc}
\usepackage[usenames, dvipsnames]{color}
\definecolor{sdublue}{cmyk}{1.00,0.65,0.0,0.3}
\usepackage{lastpage}
\usepackage{fancyhdr}
\pagestyle{fancy}
\chead{\Large \textcolor{sdublue}{\bfseries\rmfamily Handlingsplan for kursus under Det Naturvidenskabelige Studienævn}}
\renewcommand{\headrulewidth}{0pt}
\cfoot{Page \thepage\ of \pageref{LastPage}}

\usepackage{tabu}

\newcommand{\TODO}[1]{\begingroup\color{red}#1\endgroup}
\newcommand{\answer}{\textcolor{red}{Please answer}}

\setlength\parindent{0pt}
\begin{document}
\tabulinesep=1.2mm
\begin{tabu} to 1.0\textwidth { | >{\mbox{}\color{sdublue}\bf}X[l] | X[2l] | } \hline
Kursuskode og navn &  DM561 --- Linear Algebra with Applications \\\hline
Semester {\normalfont (Efterår/forår og årstal)} & Efterår 2018 \\\hline
Undervisnings\-ansvarlig &  Marco Chiarandini (sammen med Christian Kudahl, Daniel Merkle)  \\\hline
Øvrige undervisere &   Jonas Herskind Sejr (instructor) \\\hline
Antal tilmeldte studerende til kurset & 65 in BlackBoard, 63 in the protocol \\\hline
Antal studerende, som har deltaget i evalueringen &  8  \hfill
Svarprocent:ca.~50\% \\\hline
På hvilke studieretninger og semestre indgår kurset &  The course is elective in the second year of the Physics and Chemistry curriculum. In addition, this year there were three exchange Computer Science students and one Applied Mathematics student \\\hline
Hvilken evalueringsform har været anvendt &  classical eletronic questionnaire \\\hline
Har der været foretaget en midtvejsevaluering på kurset? &  yes, with the Delphi method  \\ 
\hline
\end{tabu}

~\\[1cm]\textcolor{sdublue}{{\bf Beskriv evalueringens resultater} ---
  f.eks. indenfor:\\ {\small \em Kursets opbygning og placering,
    emner, undervisningsformer, fordeling af øvelsestimer,
    forelæsninger og e-timer mv, de studerendes arbejdsbelastning,
    undervisningsmaterialet, delprøver og forudsætningsprøver,
    sammenhæng med studiet, de enkelte læreres undervisning:}}


65 52

More than 85\% = 510 points gets a level increase if passed or close to
passed (ie, 18+) in first part



The course ran this year for the second time. It gives 5 ECTS and it is
supposed to be an introductory course in programming assuming no
previous knowledge. The course was designed to be an elective course for
Physicist. The choice of using the programming language C++ was taken by
the FKF department. This programming language is suitable for scientific
computing due to its presumed efficiency and safeness against possible
bugs. Probably due to a flaw in the system, the course had this year
also 3 exchange students in Computer Science and an Applied Math
student.

The course run from October to December, thus it was spread more
around the semester than done the previous year.

The text book was the same as the previous year. The course followed
the text book closely and presented the programming language C++ in a
modern way, that is, leaving memory management at the last and using
vectors from the standard library as the main type to avoid issues with
memory management.  The contents of the course were shrunk removing the
part on the graphics library.

There were scheduled 14 introductory classes of two hours and 10
training sessions. In addition, there were 2 class tests that were part
of the final project and thus obligatory for the final pass/fail
assessment. I held a flipped classroom class, which was recognised as
useful by the students but few prepared for it. I used in class a
reduced set of slides with respect to those later published for the
students, in order to give more emphasis on the important aspects.  In
the training classes, students had to work at a few exercises that were
given in advance to let them prepare at home. Several activities were
planned as group work and particularly work in pairs.

The issues related with the working environment (Windows and Visual
C++), which caused several problems the previous year, were partially
improved by using command line tools (for building programs), simpler
text editors and Windows Subsystem for Linux. However, students still
claim to miss a introduction to command line and unix tools, such as
remote connection to Linux machines for exchanging the files, running
the code, etc.

The final project consisted in implementing one of four different
programs of varying complexity. The choice was left to the
students. They all ended up choosing the easiest. The intention with
alternative programs was to motivate the students and to meet the
needs for different levels.

The time allocated for carrying out the project was close to one month
after the course ended. A deadline extension was asked a few days before
the deadline and it was declined.

The course materials are available at:
http://www.imada.sdu.dk/~marco/DM560/.

\medskip
In the course evaluation via electronic questionnaire, there were 5
respondents from Physics and Chemistry, and 2 from Computer Science.
The outcome can be interpreted as follows.



%
%• The teacher was not good in understanding were the students might have difficulties.
%• The setup of the course was not in order.
%• The exercises were not well chosen.
%• The pensum was not well suited to the premises of the students and too large.
%• The introductory classes were not relevant, most of the learning occurred during the laboratory hours where students were asked to program.
%• The introductory classes were much ahead with respect to the exercises. Or the course was too fast.
%• The progression has been too steep, with a low start and then a fast increase in difficulty and amount of work.
%• The second hand in was too difficult.
%
%
%• There were good possibilities to give feedback to the teacher.
%• The structure and the expectations have been overall made clear enough and the course page was usable and satisfactory.
%• Overall the atmosphere during the course was polite and welcoming.
%• The instructor was very helpful.
%• The activities in group were helpful.
%• The laboratory classes were good.
%• The slides were good.
%• It was appreciated that the teaching form changed during the course to meet the students’ status.
%


\begin{itemize}
\item The students declare to have spent in average less than 10 hours
  per week on this course.

\item The course is perceived between average and very difficult.

\item Some students comment that the course was for Computer
  Scientists rather than for physicists due to the perception that a
  certain knowledge of programming was assumed and that more emphasis
  was given to understanding how things work rather than just making
  them work. For these reasons, the course has been perceived as
  challenging and overall not satisfactory for the final outcome.

\item Some students, presumably the Physicists, report that the reasons
  for not attending classes have been:
  \begin{itemize}
  \item overlaps with other courses in the semester;
\item high workload during the semester (apparently all 30 ECTS, 5 courses,
  were overlapping);
\item burden of going to classes without having prepared due to the
  requirement of a certain degree of active participation;
\item difficulty to recover from have lagged behind and feeling of being judged
  in class.
\end{itemize}

\item There are constantly two modes (satisfied and not satisfied) in the
distribution of students about the information available before the
course, the prerequisites, the motivations and the organization of the course. Most
likely the two modes are due to the Physics students being
dissatisfied, and the Computer Science students being satisfied.

\item The distribution of students about the relevance of the course
  and the pedagogical competences of the teacher have long tails but
  peak on dissatisfaction. The preparation, knowledge, commitment,
  respect and understanding of the teacher are assessed as
  satisfactory.

\item It was not clear what was expected from the final project. There is
disagreement about the class test. It was asked whether it gave the
opportunity to test the understanding of what was learned and to bring
one self up-to-date with the contents of the course but no clear
assessment arose. The exercises in class seem well chosen and
beneficial although the calculator example was deemed  difficult and
shorter exercises would be preferred.

\item From the general comments it arises that: more summaries and
  setting in perspective are desired.  More time should be dedicated to
  installation and preparation of the environment, including an
  introduction to linux.

\item There is a general satisfaction with the instructor.

\item The course web page outside BlackBoard is assessed as helpful.

\end{itemize}

  
~\\[1cm]\textcolor{sdublue}{{\bf Giver evalueringen anledning til justering af undervisning mv.}:\\
{\small \em Hvis ja, beskriv hvilke:}}


 \begin{itemize}
 \item The main changes from the previous edition have been recognized
   as valid by the students and they will therefore be kept. Hence,
   there will be no treatment of the graphics library and there will
   be two class tests. However the class tests will be made optional
   and used only for self assessment so as to avoid the feeling of
   being judged. They will not be linked to the final assessment.



  It will be made clear that questions of any type during classes are
  welcome.
 
   
 \item The course will be scheduled even more spread throughout the
   semester running in parallel with the other semester courses. This
   should give a slower pace which perhaps suits better with a semester
   that is tough for Physicists. A version of this course condensed in
   half semester has already been tried and it did not work. More
   lectures will be allocated in the first weeks to ensure everybody has
   a working environment up and running.

%   
%   • More lectures will be put in the beginning of the semester to
%   ensure less load on student when hand-ins in other courses kick in
%   later on. As well as to ensure that everyone has all necessary
%   programs installed on their machines as quick as possible to make
%   the tutorial sessions most productive.
%

 \item The adoption of a different text book will be considered. The
   example on the calculator that is a central part of the current text
   book is perceived as difficult.  Ideally, a new text book should be
   more concise, and give less emphasis to memory management and error
   handling. These will be in any case the aims in the organization.



 \item More time will be spent to introduce the working environment and
   ensure that the students are settled with it. An announcement will be
   sent a couple of weeks before the course, welcoming to the course and
   inviting students to prepare their computers by either installing
   recent versions of the operating systems and other tools or getting
   acquainted with the computer lab machines. Specific instructions on
   this material will be made available from the course web page.
   


 \item The contents of the course will include clear structure and will
   be available from the start of the course, so students can see what
   they will learn in the course as well as be able to ask for
   adjustments of the content during the course.

\item It will be made more clear what are the expectations from the
  final project. The description of the last year final projects will
  be made available from the start, so the students can see what they
  are expected to know. They could as well come with possible topics
  for the final projects themselves. In this way, they will be more
  interested and excited about learning.


 \item Before the class tests, examples from this year tests will be
   made available for the preparation. The questions in the class test
   will be made easier and such that only base knowledge is tested.



  %
%   • Lecture notes will be again made in two versions, one for
%   lectures in class and one for the student (maybe two versions
%   available for students).
%

 \item The slides will be restructured possibly making a condensed
   version of them in order to make it easier to find content in
   preparation to classes and during practice.
%
%   
%   • Lecture notes for students will be restructured to be
%   self-contained (easy for self-study) with minimum important
%   information on key topics of the lectures, so that they can quickly
%   look through them before the lectures (ensuring everyone have time
%   for this) and easily consult with them later on if necessary (as
%   exam preparation e.g.)
%
%
%   •	Good online video lectures/course to supplement the course, with references to certain videos for the corresponding lectures.
%

   
%   • To ensure the students get most out of the course, the contents
%   of the course is adjusted to include the most important topics in
%   order to teach one basic programing skills. More advanced topics,
%   which contribute to deeper understanding of the C++ language, will
%   be reviewed rather introductorily supplemented with references to
%   the related literature/videos for self-study.
%

 \item In the introductory classes live coding will be used and the
   whole work flow for building a program shown several times. It arose
   from the final project that there were still lacks in this
   fundamental step.
   
 \item The training sessions will contain shorter and easier exercises
   to start with. Pair work will still be recommended as a work mode.
%  
%   
%   • To make sure all students learn how to program to the minimum
%   extent in C++, the tutorial exercises are chosen in a way that the
%   key concepts of the lectures are exemplified with small
%   exercises/programs, which are manageable and clear for all
%   levels. The small/easy exercises are accompanied by at least one
%   bigger and more advanced exercise on the lecture topic to meet the
%   advanced level students.
   %


%   
%   • Live coding will be a big part of the lectures, i.e. an
%   introduction of a new concept and right away a quick small live
%   coding example on how to make it work (thus giving an example of
%   the whole work flow).
%
   
%   • Midterm exam questions will be adjusted for the people actually
%   taking the course (e.g. physicists, mathematicians, etc., not CS
%   students) and will include the topics reflecting the key
%   concepts/features of the C++ language, which were covered during
%   the lectures to the date. These questions/answers should show how
%   well the student understood the material and whether the basic
%   knowledge were obtained.
%  

%   •	Both lecturer and instructor will try to be more pedagogical in explanation of the material; will try to make sure everyone understands the key moments and important points of the programming language and actually learn how to program; will try to keep up the warm and trustworthy atmosphere in order to encourage the students to ask questions, etc.
%
%   •	Towards the end of the course the students can be asked about any additional topics they want to be covered or any of the past topics repeated or elaborate more upon.
%
%   •	Not sure about group work (work in pair), will write like this: Students will be encouraged to work in groups/pairs by encouraging them to help each other if some of the classmates are stuck on something. This will help the students to get to know each other (if they are not from the same class/education) and will help them to progress faster and learn more.
%
%

 \end{itemize}


~\\[1cm]\textcolor{sdublue}{{\bf Giver evalueringen anledning til ændring i kursusbeskrivelsen?}\\
{\small \em Hvis ja; beskriv hvilke:}}

Nej. 

~\\[1cm]
\begin{tabu} to 1.0\textwidth { | >{\bf}X[3l] | >{\bf}X[l] | }
  \hline
  Behandlet af undervisningsudvalget på: & Dato:\\ & \\ & \\ \hline
\end{tabu}
\end{document}


%%% Local Variables:
%%% mode: latex
%%% TeX-master: t
%%% End:
