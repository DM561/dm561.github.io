% latex handlingsplan template v0.001, daniel
\documentclass[11pt]{article}
\usepackage[a4paper, total={7in, 9in}]{geometry}
\usepackage{helvet}
\renewcommand{\familydefault}{\sfdefault}
\usepackage[UKenglish]{babel}
\usepackage[utf8]{inputenc}
\usepackage[usenames, dvipsnames]{color}
\definecolor{sdublue}{cmyk}{1.00,0.65,0.0,0.3}
\usepackage{lastpage}
\usepackage{fancyhdr}
\usepackage{url}
\pagestyle{fancy}
\chead{\Large \textcolor{sdublue}{\bfseries\rmfamily Handlingsplan for kursus under Det Naturvidenskabelige Studienævn}}
\renewcommand{\headrulewidth}{0pt}
\cfoot{Page \thepage\ of \pageref{LastPage}}

\usepackage{tabu}

\newcommand{\TODO}[1]{\begingroup\color{red}#1\endgroup}
\newcommand{\answer}{\textcolor{red}{Please answer}}

\setlength\parindent{0pt}
\begin{document}
\tabulinesep=1.2mm
\begin{tabu} to 1.0\textwidth { | >{\mbox{}\color{sdublue}\bf}X[l] | X[2l] | } \hline
Kursuskode og navn &  DM561 --- Linear Algebra with Applications \\\hline
Semester {\normalfont (Efterår/forår og årstal)} & Efterår 2018 \\\hline
Undervisnings\-ansvarlig &  Marco Chiarandini (sammen med Christian Kudahl, Daniel Merkle)  \\\hline
Øvrige undervisere &   Jonas Herskind Sejr (instructor) \\\hline
Antal tilmeldte studerende til kurset & 65 in BlackBoard, 63 in the protocol \\\hline
Antal studerende, som har deltaget i evalueringen &  15  \hfill
Svarprocent:ca.~23\% \\\hline
På hvilke studieretninger og semestre indgår kurset &  The course is mandatory in the second year of the Computer Science curriculum. \\\hline
Hvilken evalueringsform har været anvendt &  eletronic questionnaire \\\hline
Har der været foretaget en midtvejsevaluering på kurset? &  yes  \\ 
\hline
\end{tabu}

~\\[1cm]\textcolor{sdublue}{{\bf Beskriv evalueringens resultater} ---
  f.eks. indenfor:\\ {\small \em Kursets opbygning og placering,
    emner, undervisningsformer, fordeling af øvelsestimer,
    forelæsninger og e-timer mv, de studerendes arbejdsbelastning,
    undervisningsmaterialet, delprøver og forudsætningsprøver,
    sammenhæng med studiet, de enkelte læreres undervisning:}}




The course ran this year for the first time. It gives 10 ECTS and it
is made of two parts, a first part shared with MM505, Linear Algebra,
taught by Christian Kudahl, where students learn the theory of linear
algebra, and a second part, taught by Daniel and Marco, where students
learn to use linear algebra in practical applications and to program
in Python.

The final grade was based on: a theoretical assignment from the first
part; and six small programming assignments from the second part. The
theoretical assignment was graded, the practical assignments
distributed a score. Both assignments must have passed to pass the
course. The final grade was decided to be the one from the theoretical
assignment increased by one level if the score in the practical
assignment was more than 85\% of the total score.

The practical assignments were conducted through a submission system
that was providing automated grading on hourly basis with possibility
to resubmit until the deadline.

The students who passed the course in this way were 52.

The course material for the second part is available at
\url{https://dm561.github.io/}.




\medskip

From the electronic questionnaire the following evaluation arises.

\begin{itemize}
\item Students have spent in average less than 10 hours or 11-15 hours per week on this course.

\item The course is perceived as:

\begin{itemize}
\item  easy or average  in comparison to others
\item  coherent with the study curriculum
\item matching the information available beforehand
\end{itemize}

\item The workload is considered at average level.

\item The planning of the course was considered satisfactory.
\item The teaching material was useful and relevant.

\item The second part of the course is perceived as:
 helpful to improve programming skills,
and to understand the importance of Linear Algebra.

\item The programming assignments in the second part of the course
  definitely motivated students in their study. The results provided on
  hourly basis were used to improve the code submitted.

\item All assignments of the second part were found relevant for the
  course.

\item There was no criticism to the three teachers in the open field
  for this in the questionnaire. The only issue seems to be a lack of
  coordination with other courses of the semester in regards to the
  deadlines for the assignments.

  The were positive comments encouraging to preserve the course as it
  is.
  
\item The instructor was evaluated positively under all aspects: level
  of knowledge, pedagogical competences, preparation, commitment.


  
\end{itemize}




  
~\\[1cm]\textcolor{sdublue}{{\bf Giver evalueringen anledning til justering af undervisning mv.}:\\
{\small \em Hvis ja, beskriv hvilke:}}

No, we do not plan any change.


~\\[1cm]\textcolor{sdublue}{{\bf Giver evalueringen anledning til ændring i kursusbeskrivelsen?}\\
{\small \em Hvis ja; beskriv hvilke:}}

No. We do not plan any change.

~\\[1cm]
\begin{tabu} to 1.0\textwidth { | >{\bf}X[3l] | >{\bf}X[l] | }
  \hline
  Behandlet af undervisningsudvalget på: & Dato:\\ & \\ & \\ \hline
\end{tabu}
\end{document}


%%% Local Variables:
%%% mode: latex
%%% TeX-master: t
%%% End:
