\documentclass[a4paper,10pt]{article}
%\pdfoutput=1
\usepackage[utf8]{inputenc}
\usepackage[english]{babel}
\usepackage{graphicx}
\usepackage[a4paper,margin=2.5cm]{geometry}
%\usepackage{palatino}
\usepackage[math]{kurier}
%\usepackage[sc]{mathpazo} % consider options: osf, sc
\usepackage{xspace}
%\PassOptionsToPackage{latex2html}{hyperref}  
%\usepackage{html} 
\usepackage{url} 
%\usepackage{booktabs} 
%\usepackage{hyperref}
%\usepackage{latexsym}
\usepackage{amsmath}
\usepackage{amssymb}
\usepackage{amsfonts}
\usepackage{rotating}
%\usepackage{srctex}
\usepackage{color}
\usepackage{enumerate}
%\usepackage{afterpage}

\usepackage{arydshln}
\usepackage{etoolbox}
\usepackage{blkarray}
\usepackage{multirow}

\usepackage{listings}
\usepackage[vlined]{algorithm2e}



\usepackage{fancyhdr}
\pagestyle{fancy} \lhead{{DM561 -- {\sc Autumn} 2021}} \chead{}
\rhead{{\sc Assignment}}

%\setlength{\headheight}{4ex}
\fancypagestyle{plain}{
\lhead{Department of Mathematics and Computer Science\\
University of Southern Denmark, Odense}
\chead{}
\rhead{\today\\
DM \& MC}
\lfoot{}
\cfoot{\thepage}
\rfoot{}
\renewcommand{\headrulewidth}{0pt}
}


\usepackage{fancyvrb}
\RecustomVerbatimEnvironment{Verbatim}{Verbatim}{xleftmargin=5mm}

\graphicspath{{Figures/}}




%%%%%%%%%%%%%%%%%%%%%%%%%%%%%%%%%%%%%%%%%%%%%%%%%%%%%%%%%%%%
\usepackage{mylstset}
\usepackage{mystyle}
%%%%%%%%%%%%%%%%%%%%%%%%%%%%%%%%%%%%%%%%%%%%%%%%%%%%%%%%%%%%
\newif\ifshow
%\showtrue
\showfalse
\usepackage{exercises}
%%%%%%%%%%%%%%%%%%%%%%%%%%%%%%%%%%%%%%%%%%%%%%%%%%%%%%%%%%%%
\usepackage{tikz}
\usetikzlibrary{arrows,automata}
\usepackage[all]{xy}


\newcommand{\row}[1]{
%HEVEA \begin{comment}
\ensuremath \overrightarrow{\mathbf #1}
%HEVEA \end{comment}
%HEVEA \overrightarrow{\mathbf #1}
}



\graphicspath{{Figures/}}


\title{%
\begin{flushleft}
  DM561 -- Linear Algebra and Applications\\
  DM562 -- Scientific Programming\\[0.3cm]
{\Large Obligatory Assignment 2, Autumn 2021} %
% \ahref{\url{http://www.imada.sdu.dk/~marco/DM559/Assignments-LA/sheet2.pdf}}{\tiny [pdf format]}
\\
\hrulefill
\\[-1.8cm]
\end{flushleft}
}


%\hrulefill \\[-1.8cm]
\author{}
\date{}


\begin{document}

\maketitle



\begin{solution}
Contains Solutions!
\end{solution}



\begin{center}
%\color{red}
  {\bf Deadline: Monday, November 23 at noon}
%\color{black}
\end{center}

%\color{red}In red the modifications after publication.\color{black}


This document is associated with the files \lstinline{asg2.py} and
\lstinline{draw.py} and data files, which are available in the git
repository. The file \lstinline{asg2.py} is the only one that needs to be
edited and submitted.

\bigskip

%
%In \lstinline{asg2.py}, to make sure your code passes the
%docstring example tests you have to add
%\begin{lstlisting}
%np.set_printoptions(precision=3)
%\end{lstlisting}
%
%\vspace{-3ex}
%right after the import of Numpy.
%
%In  \lstinline{draw.py} you have to change the line:
%
%\begin{lstlisting}
%from asg2_sol import * 
%\end{lstlisting}
%
%\vspace{-3ex}
%
%to
%\begin{lstlisting}
%from asg2 import * 
%\end{lstlisting}
%\color{black}
%
%
%\bigskip
%\subsection*{Task 1}

In the Introduction to Python - Part 3, on slide 12, we compared three
important models for growth functions in computer science applications:

\begin{table}[h]
  \begin{tabular}{ll}
    \textbf{exponential model}&$y=ae^{bx}$\\
    \textbf{power function model}&$y=ax^b$\\
    \textbf{logarithmic model}&$y=a+b\ln x$
  \end{tabular}
\end{table}

where $a$ and $b$ are to be determined to fit experimental data as
closely as possible. In this exercise you will work with a procedure
called \emph{linearization}, by which the data are transformed to a form
in which a least squares straight line fit can be used to approximate
the constants. % Some calculus is required to carry out the task.


Let $x$ denote the different size of two square matrices that we
multiply with each other and $y$ the computation time registered by
executing, for example, the script benchmark from Assignment 1.

We will assume to have collected the following data $D=\{(x_i,y_i)\}$
(also available in the associated python script):

\begin{table}[h]
  \centering
  \begin{tabular}{l|*{9}{c}}
    x&2&3&4&5&6&7&8&9\\
    y&1.75&1.91&2.03&2.13&2.22&2.30&2.37&2.43
  \end{tabular}
\end{table}
    
We will fit a linear model and the three models above using least
squares and decide which is the best model.



\paragraph{Linear function} Implement the function
\lstinline{least_squares(A,b)} that takes an appropriate matrix $A$ and
vector $\vec b$ and returns the least square solution $\vec z$ of the
system $A\vec z=\vec b$. In the implementation, you are not allowed to
use the function function \lstinline{linalg.lstsq(A,b)} from Numpy and
Scipy.  Using the function \lstinline{least_squares(A,b)} determine the
linear function $y=ax + b$ that best fits the given data $(x_i,y_i)$.
You can visualize the situation executing the file \lstinline{draw.py}
that uses the code of the function \lstinline{linear_model(x,y)} that
you implement to draw a plot with the points and the fitted linear
regression.

\paragraph{Exponential function}
Making the substitution \[Y=\ln y\] in the equation $y=ae^{bx}$ produces the
equation $Y=bx+\ln a$, whose graph in the $xY$-plane is a line of slope
$b$ and $Y$-intercept $\ln a$.  Verify this fact!

Hence, a curve of the form $y=ae^{bx}$ can be fitted to the given $n$ data
points $(x_i,y_i)$ by letting $Y_i=\ln y_i$, then fitting a straight
line to the transformed data points $(x_i,Y_i)$ by least squares to find
$b$ and $\ln a$, and then computing $a$ from $\ln a$.  Implement this
method to fit an exponential model in the function
\lstinline{exponential_fit} that uses your
\lstinline{least_squares(A,b)}. (You may need to use the functions
\lstinline{np.log} and \lstinline{np.exp} for computing the natural
logarithm). You can visualize the situation executing the file
\lstinline{draw.py} that uses the function that you implement to draw a
plot with the points and the fitted exponential curve.

\paragraph{Power function}
Making the substitutions \[X=\ln x\qquad \qquad Y=\ln y\] in the
equation $y=ax^b$ produces the equation $Y=bX+\ln a$, whose graph in the
$XY$-plane is a line of slope $b$ and $Y$-intercept $\ln a$.  Verify
this fact!

Hence, a curve of the form $y=ax^b$ can be fitted to the given $n$ data
points $(x_i,y_i)$ by letting $X_i=\ln x_i$ and $Y_i=\ln y_i$, then
fitting a straight line to the transformed data points $(X_i,Y_i)$ by
least squares to find $b$ and $\ln a$, and finally computing $a$ from $\ln
a$.  Implement this method to fit an power model in the function
\lstinline{power_fit}. You can visualize the situation executing the
file \lstinline{draw.py} that uses the function that you implement to
draw a plot with the points and the fitted curve.


\paragraph{Logarithmic function} Making the substitution \[X=\ln x\]
in the equation $y=a+b\ln x$ produces the equation $Y=a+bX$, whose
graph in the $Xy$-plane is a line of slope $b$ and $y$-intercept
$a$. Verify this fact!

Hence, a curve of the form $y=a+b\ln x$ can be fitted to the given $n$ data
points $(x_i,y_i)$ by letting $X_i=\ln x_i$ and then fitting a straight
line to the transformed data points $(X_i,y_i)$ by least squares to find
$b$ and $a$.  Implement this
method to fit a logarithmic model in the function
\lstinline{logarithmic_fit}. You can visualize the situation executing
the file \lstinline{draw.py} that uses the function that you implement
to draw a plot with the points and the fitted curve.


\paragraph{Training error} Implement the function
\lstinline{training_error(f,x,y)} that returns the \color{red}
{\bf root}\color{black}\ sum of squared
  errors for the model \lstinline{f}. Using this function compare the
  training error of the four models and determine which model has the
  best training error.



\begin{figure}[tb]
\centering
\includegraphics[width=\textwidth]{curves1}
\includegraphics[width=\textwidth]{curves2}
\caption{The result of this assignment with training error on two
  different data sets. Which curve fits best the data in the two data sets? }
\end{figure}



\end{document}
\clearpage
  



%%%%%%%%%%%%%%%%%%%%%%%%%%%%%%%%%%%%%%%%%%%%%%%%%%%%%%%%%%%%%%%%%%%%%%%%
%%%%%%%%%%%%%%%%%%%%%%%%%%%%%%%%%%%%%%%%%%%%%%%%%%%%%%%%%%%%%%%%%%%%%%%%
%%%%%%%%%%%%%%%%%%%%%%%%%%%%%%%%%%%%%%%%%%%%%%%%%%%%%%%%%%%%%%%%%%%%%%%% 
%%%%%%%%%%%%%%%%%%%%%%%%%%%%%%%%%%%%%%%%%%%%%%%%%%%%%%%%%%%%%%%%%%%%%%%%



\end{document}
