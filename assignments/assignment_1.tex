\documentclass[a4paper,10pt]{article}
%\pdfoutput=1
\usepackage[utf8]{inputenc}
\usepackage[english]{babel}
\usepackage{graphicx}
\usepackage[a4paper,margin=2.5cm]{geometry}
%\usepackage{palatino}
\usepackage[math]{kurier}
%\usepackage[sc]{mathpazo} % consider options: osf, sc
\usepackage{xspace}
%\PassOptionsToPackage{latex2html}{hyperref}  
%\usepackage{html} 
\usepackage{url} 
%\usepackage{booktabs} 
%\usepackage{hyperref}
%\usepackage{latexsym}
\usepackage{amsmath}
\usepackage{amssymb}
\usepackage{amsfonts}
\usepackage{rotating}
%\usepackage{srctex}
\usepackage{color}
\usepackage{enumerate}
%\usepackage{afterpage}

\usepackage{arydshln}
\usepackage{etoolbox}
\usepackage{blkarray}
\usepackage{multirow}

\usepackage{listings}
\usepackage[vlined]{algorithm2e}


\usepackage{fancyvrb}
\RecustomVerbatimEnvironment{Verbatim}{Verbatim}{xleftmargin=5mm}

\graphicspath{{Figures/}}




%%%%%%%%%%%%%%%%%%%%%%%%%%%%%%%%%%%%%%%%%%%%%%%%%%%%%%%%%%%%
\usepackage{mylstset}
\usepackage{mystyle}
%%%%%%%%%%%%%%%%%%%%%%%%%%%%%%%%%%%%%%%%%%%%%%%%%%%%%%%%%%%%
\newif\ifshow
\showtrue
%\showfalse
\usepackage{exercises}
%%%%%%%%%%%%%%%%%%%%%%%%%%%%%%%%%%%%%%%%%%%%%%%%%%%%%%%%%%%%
\usepackage{tikz}
\usetikzlibrary{arrows,automata}
\usepackage[all]{xy}


\newcommand{\row}[1]{
%HEVEA \begin{comment}
\ensuremath \overrightarrow{\mathbf #1}
%HEVEA \end{comment}
%HEVEA \overrightarrow{\mathbf #1}
}



\graphicspath{{Figures/}}


\title{%
\begin{flushleft}
DM561 -- Linear Algebra and Applications\\[0.3cm]
{\Large Obligatory Assignment 1, Autumn 2018} %
% \ahref{\url{http://www.imada.sdu.dk/~marco/DM559/Assignments-LA/sheet2.pdf}}{\tiny [pdf format]}
\\
\hrulefill
\\[-1.8cm]
\end{flushleft}
}


%\hrulefill \\[-1.8cm]
\author{}
\date{}


\begin{document}

\maketitle



\begin{solution}
Contains Solutions!
\end{solution}



\begin{center}
\color{red}
  {\bf Deadline: Wednesday, April 4 at 18:00.}
\color{black}
\end{center}

\color{red}In red the modifications after publication.\color{black}

This is the second obligatory assignment in DM559. The number 0.2
indicates that this is Assignment 2 of the part 0 of the course (the
part on Linear Algebra). The part 1 will start in week 12.

The Assignment has to be carried out individually.  You can consult with
peers, if you are unable to proceed, but without exchanging full
solutions.


The submission is electronic via:
\begin{center}
\url{http://valkyrien.imada.sdu.dk/milpApp/}.
\end{center} 
%The submission is individual and
%electronic via Blackboard SDU Assignment.



The deliverable is a PDF document, as it will be at the written
exam. Hence, experiment and get acquainted with the tools and forms you
want to use for producing this document in a similar setting as the
exam. In any case, you have to put your answers in this
\myhref{\mypage/Assignments-LA/template_answers.tex}{template} You can
handwrite your answers and add them as picture in the template. Be aware
that at the exam you can use digital pen or hand scanner (that is, a
silent scanner) but you cannot bring handycameras. You can of course
also typset your answers in LaTeX. You can use either Danish or English.
Keep the newpage separation after each
\color{red}exercise\color{black}\ present in the template. Write your
name and CPR number where indicated in the template.

\medskip

In some parts, you are asked to write Python code. You have to include
the code in the PDF document \emph{together with the output of the
  execution of the code}. Use the LaTeX environment \verb=lstlisting=
as shown in the template for doing this. Your code must work correctly
if fully copied and pasted from your report.
%
%Only PDF files are accepted. Use the this Latex template:
%\url{http://www.imada.sdu.dk/~marco/DM559/Files/template_assignments.tex}
%Handwriting the answers and including a picture or a scan of them in the
%template is allowed.
%

\medskip
In your answers, you have to justify the steps that you are doing. If
theorems and definitions from the slides of the course or from the text
book (specify which edition) are used, give reference.


\medskip
The tasks are all about linear algebra. Your goal is to answer correctly
to as many subtasks as you can. Tasks and subtasks are presented in
increasing order of difficulty. You are welcome to ask in class for
explanations that could put you on the right path for solving the tasks.

\medskip
\emph{Exercises \color{red}1-3b\color{black}\ must be completed to pass the assignment and I
  expect that they can be carried out in less than one working day if
  you are up-to-date with the subject. Exercises \color{red}3c-5\color{black}\ are real-life
  applications of linear algebra.  }



\clearpage


\exercise{*}{Change of basis}





\end{document}
