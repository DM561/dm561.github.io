%\documentclass[a4paper,10pt]{report}
\documentclass[a4paper,10pt]{article}
%\pdfoutput=1
\usepackage[utf8]{inputenc}
\usepackage[english]{babel}
\usepackage{graphicx}
\usepackage[a4paper,margin=2.5cm]{geometry}
%\usepackage{palatino}
\usepackage[math]{kurier}
%\usepackage[sc]{mathpazo} % consider options: osf, sc
\usepackage{xspace}
%\PassOptionsToPackage{latex2html}{hyperref}  
%\usepackage{html} 
\usepackage{url} 
%\usepackage{booktabs} 
%\usepackage{hyperref}
%\usepackage{latexsym}
\usepackage{amsmath}
\usepackage{amssymb}
\usepackage{amsfonts}
\usepackage{rotating}
%\usepackage{srctex}
\usepackage{color}
\usepackage{enumerate}
%\usepackage{afterpage}

\usepackage{arydshln}
\usepackage{etoolbox}
\usepackage{blkarray}
\usepackage{multirow}

\usepackage{listings}
\usepackage[vlined]{algorithm2e}


\usepackage{fancyvrb}
\RecustomVerbatimEnvironment{Verbatim}{Verbatim}{xleftmargin=5mm}

\graphicspath{{Figures/}}

\usepackage{physics}


%%%%%%%%%%%%%%%%%%%%%%%%%%%%%%%%%%%%%%%%%%%%%%%%%%%%%%%%%%%%
\usepackage{mylstset}
\usepackage{mystyle}
%%%%%%%%%%%%%%%%%%%%%%%%%%%%%%%%%%%%%%%%%%%%%%%%%%%%%%%%%%%%
\newif\ifshow
\showtrue
%\showfalse
\usepackage{exercises}
%%%%%%%%%%%%%%%%%%%%%%%%%%%%%%%%%%%%%%%%%%%%%%%%%%%%%%%%%%%%
\usepackage{tikz}
\usetikzlibrary{arrows,automata}
\usepackage[all]{xy}


\newcommand{\row}[1]{
%HEVEA \begin{comment}
\ensuremath \overrightarrow{\mathbf #1}
%HEVEA \end{comment}
%HEVEA \overrightarrow{\mathbf #1}
}



%%%%%%%%%%%%%%%%%%%%%%%%%%%%%%%%%%%%%%%%%%%%%%%%%%%%%%%%%%%%
\usepackage{enumitem}
\newcommand{\lastwords}{End of Examination}
\newlist{subquestion}{enumerate}{1}
\setlist[subquestion,1]{label=(\alph*)}
\def\la{\langle}
\def\ra{\rangle}
\def\R{\mathbb R}
%%%%%%%%%%%%%%%%%%%%%%%%%%%%%%%%%%%%%%%%%%%%%%%%%%%%%%%%%%%%



\setcounter{tocdepth}{1}




\title{%
\begin{flushleft}
DM561 -- Linear Algebra with Applications\\[0.3cm]
%{\Large Reexam for Obligatory Assignments, Autumn 2018} %
%\ahref{\url{http://www.imada.sdu.dk/~marco/DM559/Assignments-LA/sheet2.pdf}}{\tiny
%  [pdf format]}
%\\
\hrulefill
\\[-1.8cm]
\end{flushleft}
}


%\hrulefill \\[-1.8cm]
\author{}
\date{}


\begin{document}

\maketitle




\medskip

\begin{itemize}

\itemsep=1ex

\item The reexam consists of two separated assignments:

\begin{itemize}

\item \emph{Theory assignment}:
  \begin{itemize}
\item Specifications in this document
\item Hand in PDF via BlackBoard SDU Assignments.
  \end{itemize}
  
\item \emph{Applications assignment}:
  \begin{itemize}
  \item Specifications at the course web page
    \url{https://dm561.github.io/}:\\ assignments: asg1, asg2, asg3, asg4,
    asg5, asg6
%  \item {\bf Deadline: Monday, April 1, 2019 at noon}
  \item Hand in via git push as during the course.
  \end{itemize}
\end{itemize}
  
\item {\bf Deadline for both assignments: Tuesday, March 31, 2020 at noon}

\item You are expected to work individually on the assignments.

  
\item Deadline extensions will not be conceded.
\end{itemize}


\clearpage

\section*{Theory Assignment}


For questions, you may contact the lecturer by e-mail
\url{marco@imada.sdu.dk}. \\ You may answer in Danish or English. Your
solution must be handed in as a single PDF file clearly stating your
{\bf full name and birth date} on the front page. You have to hand in
using SDU Assignment before Tuesday, March 31 at noon.
%In case of
%irrecoverable technical problems with Blackboard, you may hand in the
%exam electronically by sending an email to the lecturer before the
%deadline. \\


\bigskip
% End title page

You may use the course book, the slides from the lectures, and notes from
the lectures and the exercises. You are allowed to use tools such as
Python to assist you in calculations, but you must write the calculations out
in full so that they can be followed by the examiners.  It is not
sufficient to present an answer, you must show how you found it.  If you
use a theorem from the book (or the slides), make a reference to
it. \newline
\vspace{10mm}


\noindent
{\bf Problem 1.}

\vspace{3mm}\noindent
Consider the following system of linear equations in variables $x_1$, $x_2$, $x_3$, $x_4$

$$
\begin{aligned}
x_1 + x_2 - x_4 & = 1 \\
2x_1 + x_2 +2x_3 & = 0 \\
x_2 + x_3 & = -1 \\
x_1 - x_2 + x_3 + x_4 & = 0
\end{aligned} 
$$

\vspace{1mm}\noindent
(a) Write the augmented matrix of the system. 

\vspace{1mm}\noindent
(b) Reduce the matrix to row echelon form by performing a sequence of elementary row operations. 

\vspace{1mm}\noindent
(c) Solve the system and write its general solution in parametric form. 

\vspace{20mm}\noindent
{\bf Problem 2.}

\vspace{3mm}\noindent
Consider the following matrix 
$$
A = \left[ \begin{array}{rrr} 1 & 1 & 1 \\ -1 & 1 & -2 \\ 0 & 2 & -2 \end{array} \right]
$$

\vspace{1mm}\noindent
(a) Find its inverse (if it exists) by performing suitable row operations on the double matrix 
$[A \mid I]$. 

\vspace{1mm}\noindent
(b) Calculate the determinant of $A$ by a cofactor expansion. 

\vspace{20mm}\noindent
{\bf Problem 3.}

\vspace{3mm}\noindent
Let $P_2=\{ a_0 + a_1x + a_2x^2 \mid a_0,a_1,a_2\in \R\}$ be the vector space of polynomials 
with real coefficients and of degree at most 2. Consider elements $f_1=1 + x^2$, $f_2= 1+x$, $f_3= x+x^2$ 
of this vector space. 

\vspace{1mm}\noindent
(a) Show that $\{f_1,f_2,f_3\}$ is a basis of $P_2$. 

\vspace{1mm}\noindent
(b) Write an arbitrary element $a_0 + a_1x + a_2x^2$ of $P_2$ as a linear combination of $f_1$, $f_2$ and $f_3$. 
That is, find the coordinates of $a_0 + a_1x + a_2x^2$ with respect to the basis $\{f_1,f_2,f_3\}$. 


\vspace{20mm}
\newpage
\noindent
{\bf Problem 4.}

\vspace{3mm}\noindent
Consider a linear map $T:\R^2\to\R^2$ such that 
$$
T(x,y) = (x-2y, x+y). 
$$
Find the matrix of $T$ with respect to the basis $B=\{(1,-1),(1,0)\}$ of $\R^2$. That is, find $[T]_{B,B}$. 

\vspace{20mm}\noindent
{\bf Problem 5.}

\vspace{3mm}\noindent
Consider a linear map $:\R^3\to\R^3$ such that 
$$
U:\left( \begin{array}{c} x \\ y \\ z \end{array} \right) \mapsto 
\left[ \begin{array}{rcc} 1 & 1 & 2 \\ 0 & 1 & 1 \\ -1 & 1 & 0 \end{array} \right] 
\left( \begin{array}{c} x \\ y \\ z \end{array} \right)
$$

\vspace{1mm}\noindent
(a) Find a basis for $\ker(U)$. 

\vspace{1mm}\noindent
(b) Give the rank and the nullity of $U$. 

\vspace{1mm}\noindent
(c) Doeas the vector 
$$
\left( \begin{array}{r} 2 \\ 1 \\ -1 \end{array} \right)
$$
belong to the range of $U$ (i.e. to the column space of $U$)?

\noindent
{\bf Problem 6.}

\vspace{3mm}\noindent
Consider the following matrix
$$
A=\left[ \begin{array}{rcc} 1 & 1 & 1 \\ 1 & 1 & 1 \\ 1 & 1 & 1 \end{array} \right] 
$$

\vspace{1mm}\noindent
(a) Solve the characteristic equation for $A$ and thus find its eigenvalues. 

\vspace{1mm}\noindent
(b) For each eigenvalue, find a basis of the corresponding eigenspace. 

\vspace{1mm}\noindent
(c) By the Gramm-Schmidt orthogonalization process or otherwise, find an orthonormal basis of $\R^3 $
(equipped with the standard Euclidean inner product) consisting of eigenvectors for $A$. 

\vspace{1mm}\noindent
(d) Find an orthogonal matrix $P$ and a diagonal matrix $D$ such that 
$$
A=PDP^{-1}. 
$$

\vspace{1mm}\noindent
(e) Find matrix $A^{2020}$. 

\vspace{20mm}\noindent
{\bf Problem 7.}

\vspace{3mm}\noindent
Consider the space $\R^3$ equipped with the standard Euclidean inner product, and its subspace 
$$
W = \{(x-y,x,x+y) \mid x,y\in\R\}. 
$$

\vspace{1mm}\noindent
(a) Find an orthonormal basis for $W$. 

\vspace{1mm}\noindent
(b) Let $v=(1,0,0)$. Find the orthogonal projection of $v$ onto $W$. 

\vspace{1mm}\noindent
(c) Find the distance from the vector $v$ to the subspace $W$. 

\end{document}
