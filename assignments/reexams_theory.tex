% ANUfinalexam.tex (Version 2.0)
% ===============================================================================
% Australian National University Final Exam LaTeX template.
% 2004; 2009, Timothy Kam, ANU School of Economics
% Licence type: Free as defined in the GNU General Public Licence: http://www.gnu.org/licenses/gpl.html

\documentclass[a4paper,12pt,fleqn]{article}
\usepackage{amsmath}
\usepackage{fancyhdr}
\usepackage{verbatim}
\usepackage{enumitem}
\usepackage{caption}
\usepackage{graphicx}
\usepackage{amscd, amssymb,amsmath,color}

% Insert your course information here %%%%%%%%%%%%%%%%%%%%%%%%%%%%%%%%%%

\newcommand{\institution}{University of Southern Denmark}
\newcommand{\titlehd}{Linear Algebra}
\newcommand{\examdate}{Spring 2019}
\newcommand{\examcode}{MM505 Re-exam}
\newcommand{\readtime}{3 Weeks}
\newcommand{\lastwords}{End of Examination}
\newlist{subquestion}{enumerate}{1}
\setlist[subquestion,1]{label=(\alph*)}
\def\la{\langle}
\def\ra{\rangle}

%%%%%%%%%%%%%%%%%%%%%%%%%%%%%%%%%%%%%%%%%%%%%%%%%%%%

\def\R{\mathbb R}


% ANU Exams Office mandated margins and footer style
\setlength{\topmargin}{0cm}
\setlength{\textheight}{9.25in}
\setlength{\oddsidemargin}{0.0in}
\setlength{\evensidemargin}{0.0in}
\setlength{\textwidth}{16cm}
\pagestyle{fancy}
\lhead{} 
\chead{} 
\rhead{} 
\lfoot{} 
\cfoot{\footnotesize{Page \thepage \ of \pageref{finalpage} -- \titlehd \ (\examcode)}} 
\rfoot{} 

% DEPRECATED: ANU Exams Office mandated margins and footer style
%\setlength{\topmargin}{0cm}
%\setlength{\textheight}{9.25in}
%\setlength{\oddsidemargin}{0.0in}
%\setlength{\evensidemargin}{0.0in}
%\setlength{\textwidth}{16cm}
%\pagestyle{fancy}
%\lhead{} %left of the header
%\chead{} %center of the header
%\rhead{} %right of the header
%\lfoot{} %left of the footer
%\cfoot{} %center of the footer
%\rfoot{Page \ \thepage \ of \ \pageref{finalpage} \\
%       \texttt{\examcode}} %Print the page number in the right footer

\renewcommand{\headrulewidth}{0pt} %Do not print a rule below the header
\renewcommand{\footrulewidth}{0pt}


\begin{document}

% Title page

\begin{center}
%\vspace{5cm}
\large\textbf{\institution}
\end{center}
\vspace{1cm}

\begin{center}
\textit{ \examdate}
\end{center}
\vspace{1cm}

\begin{center}
\large\textbf{\titlehd}
\end{center}

\begin{center}
\large\textbf{\examcode}
\end{center}
\vspace{4cm}


\begin{center}
You have xxx weeks to answer this take-home exam. You may use
the course book, the slides from the lecture, and notes from the
lectures and the exercises. You are allowed to use tools such as Maple to
assist in calculations, but you must write the calculations out in full so
that they can be followed by the examiners. 
It is not sufficient to present an answer, you must show how you found it.
If you use a theorem from the book (or the slides), make a reference to it. \newline
\vspace{5mm}

For questions, you may contact the lecturer by e-mail
xxx@imada.sdu.dk. \\
You may answer in Danish or English. Your solution must be handed in as
a single PDF file clearly stating your {\bf full name and birth date} on the front
page. You have to hand in using SDU Assignment before xxx.
In case of irrecoverable technical problems with Blackboard, you may
hand in the exam electronically by sending an email to the lecturer before the deadline. \\
\end{center}

% End title page



\newpage
\section*{Problem 1}
Consider the following system of linear equations in variables $x,y,z,w\in \R$.
\begin{align*}
x+2y&=0 \\
-2w+x&=4 \\
2z+2y&=-1 \\
4x+8y&=0
\end{align*}
\noindent
\begin{enumerate}
\item Write the augmented matrix of this system.
\item Reduce this matrix to row echelon form by performing a sequence of elementary row
operations.
\item Solve the system and write its general solution in parametric form.
\end{enumerate}


\section*{Problem 2}
Consider the following matrix
$$
M=\left[ \begin{array}{ccc} 1 & 0 & 1 \\ -1 & 1 & 0 \\ 2 & 2 & 2 \end{array} \right].
$$ \\
\noindent
\begin{enumerate}
\item Find $M^{-1}$ by performing row operations on the double
matrix $[M\; | \;I]$.
\item Is it possible to express $M$ as a product of elementary matrices? Explain why or why not.
\end{enumerate}

\newpage

\section*{Problem 3}
Consider the following matrix
$$
M=\left[ \begin{array}{ccc} x_1 & x_2 & x_3 \\ x_4 & x_5 & x_6 \\ x_7 & x_8 & x_9  \end{array} \right]
$$ \\
with $x_i \in \R$ for all $i$ and $\det(M)=10$.
Let $N$ be given by
$$
N=\left[ \begin{array}{ccc} x_4 & x_5 & x_6 \\ x_1 & x_2 & x_3 \\ x_7+2x_1 & x_8+2x_2 & x_9+2x_3  \end{array} \right]
$$ \\
\begin{enumerate}
\item Find $\det(N)$.
\item Find $\det(N^2)$.
\end{enumerate}



\section*{Problem 4}
Let $P_1$ be the vector space consisting of polynomials of degree at most one with real coefficients with
the usual addition and scalar multiplication and with the inner product $\langle p,q\rangle=p(0)q(0)+p(1)q(1)$.
\begin{enumerate}
\item Let $T: P_1 \to P_1$ be a linear transformation with $\text{rank}(T)=1$.
Is $S$ an isomorphism?
\item Consider the two polynomials $f,g \in P_1$ with $f(x)=x+1$ and $g(x)=2x+\frac12$.
Do these form a basis for $P_1$? If so, is it an orthonormal basis?
\item Is it possible to find two orthogonal polynomials $f,g\in P_1$ and a scalar $t\in \R$ such that 
$t \cdot f=g$? Find such an $f$, $g$ and $t$ or show that it is not possible.
\end{enumerate}

\newpage

\section*{Problem 5}
Consider the linear transformation $T: \R^2 \rightarrow \R^2$ given by
$$
	T\left( \begin{array}{c} x \\ y \end{array} \right) = \left( \begin{array}{c} 0 \\ y-3x \end{array} \right)
$$
\begin{enumerate}
\item Find a basis for the kernel of $T$.
\item Find the rank and nullity of $T$.
\item Find the matrix of T with respect to the basis $B=\left\{ \left( \begin{array}{c} 2 \\ 2 \end{array} \right), \left( \begin{array}{c} 3 \\ 0 \end{array} \right)  \right\}$.
That is, find $[T]_{B}$.
\end{enumerate}

\section*{Problem 6}
Consider the following matrix
$$
A=\left[ \begin{array}{ccc} 2 & 1 & 0 \\ 0 & 0 & 0 \\ 0 & 3 & 2 \end{array} \right]
$$ \\
\begin{enumerate}
\item Find the eigenvalues for $A$.
\item For each eigenvalue, find a basis for the corresponding eigenspace.
\item Can you find matrices $P$ and $D$ such that such that $P^*AP=D$ with $D$ diagonal? Find $D$ and $P$ or argue that it is not possible.
\end{enumerate}


\begin{center}
\vspace{3cm}
--------- \textit{\lastwords} ---------
\end{center}


\label{finalpage}

\end{document}
